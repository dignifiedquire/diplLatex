%==============================================================================
%==============================================================================
%
% LaTex Header für diplLatex - eine Diplomarbeit in Latex
% Copyright Friedel Ziegelmayer
% basierend auf einer Vorlage aus der Arbeit 'Diplomarbeit mit LaTex'
% 27. Juni 2010
% letzte Änderungen 03.08.2012
%
% Diese Datei ist Teil des gesamt Paktes diplLatex.
% Zu Verwendung in die Hauptdatei diese Datei mit %==============================================================================
%==============================================================================
%
% LaTex Header für diplLatex - eine Diplomarbeit in Latex
% Copyright Friedel Ziegelmayer
% basierend auf einer Vorlage aus der Arbeit 'Diplomarbeit mit LaTex'
% 27. Juni 2010
% letzte Änderungen 03.08.2012
%
% Diese Datei ist Teil des gesamt Paktes diplLatex.
% Zu Verwendung in die Hauptdatei diese Datei mit %==============================================================================
%==============================================================================
%
% LaTex Header für diplLatex - eine Diplomarbeit in Latex
% Copyright Friedel Ziegelmayer
% basierend auf einer Vorlage aus der Arbeit 'Diplomarbeit mit LaTex'
% 27. Juni 2010
% letzte Änderungen 03.08.2012
%
% Diese Datei ist Teil des gesamt Paktes diplLatex.
% Zu Verwendung in die Hauptdatei diese Datei mit %==============================================================================
%==============================================================================
%
% LaTex Header für diplLatex - eine Diplomarbeit in Latex
% Copyright Friedel Ziegelmayer
% basierend auf einer Vorlage aus der Arbeit 'Diplomarbeit mit LaTex'
% 27. Juni 2010
% letzte Änderungen 03.08.2012
%
% Diese Datei ist Teil des gesamt Paktes diplLatex.
% Zu Verwendung in die Hauptdatei diese Datei mit \include{header} einbinden
% 
%==============================================================================
%==============================================================================



% Definition der 'documentclass' 
% steht immer zu Beginn einer Datei

\documentclass%
[%
%  pdftex,%              PDFTex verwenden da ausschliesslich pdf-files erzeugt werden.
  a4paper,%             Verwendung von A4=Papier
  twoside,%             Zweiseitiger Druck
  11pt,%                Einstellen der Schriftgröße
  halfparskip,%         Halber Zeilenbstand zwischen Absätzen.
  ngerman,%             Deutsche Sprache
  % chapterprefix,%     Kapitel mit 'Kapitel' anschreiben.
  % bibtotocnumbered,%  Literaturverzeichnis im Inhaltsverzeichnis nummeriert einfügen.
  % idxtotoc%           Index ins Inhaltsverzeichnis einfügen.
]{scrreprt}%

%==============================================================================
%		Paketeinbindungen Schrift und Aussehen
%==============================================================================

% fontencoding festlegen (T1 i.A. die einzig sinnvolle Option für LaTex)
% d.h. dies ist die Kodierung für den Output
\usepackage[T1]{fontenc}

% inputencoding festlegen 
% (Standard ist 'latin1', unter Apple kann 'applemac' vorkommen, 'utf8' wird 
% verwendet für möglichst sauberen Austausch
% zwischen Dateisystemen)
% d.h. dies ist die Kodierung der .tex Dateien und kann durchaus varieren
\usepackage[utf8]{inputenc}

% mit diesem Paket werden die Rechschreibregeln für Wortrennung o.ä. festgelegt
% 'ngerman' entspricht der Neuen Deutschen Rechtschreibung, wohingegen 'german'
% für die Alte verwendet wird 
\usepackage[ngerman]{babel} 

% Paket zur Spezifizierung der Anführungszeichen
\usepackage[style=swiss]{csquotes}

% Benutze die Schriftfamilie 'helvet
\usepackage{helvet} 
 
% Paket zum Einstellen der Abstände zwischen den Zeilen
% Optionen sind 'singlespacing' 'onehalfspacing' und 'doublespacing'
\usepackage[onehalfspacing]{setspace} 

% Paket zum Einden von Grafiken '\includegraphics{}' und weiteren graphischen 
% Einstellungen 
\usepackage{graphicx}
\usepackage{subfigure}

 
% Pakte um Farben verwenden zu können 
\usepackage[usenames]{color} 

% Pakete um Grafiken erzeugen zu können
\usepackage{pst-plot}
\usepackage{pstricks}


% Paket zur Quellcodeformatierung in LaTex, für die Verwendung siehe die 
% Paketdokumentation 
\usepackage{listings}

% Paket zur schnellen Änderung von 'enumerate'
% Verwendung: 
% \begin{enumerate}[label=<your label>]
%	\item
%	..
%	\item
%  \end{enumerate}
%
% für <your label> schreibe die gewünschte Nummerierung wie z.B. '(\arabic*)'
% um (1), (2), usw. zu erhalten
\usepackage{enumitem}

% Paket zum erstellen eines Anhangs
\usepackage{appendix}

\usepackage{titlesec}

% Paket zur einfachen Anpassung der Kopf- und Fußzeilen
% Siehe weiter unten zur Verwendung 
\usepackage{fancyhdr}

% Paket zur genauen Defnition der einzelnen Größen im Layout des Dokuments
% siehe weiter unten '\geometry' wo die konkreten Einstellungen vorgenommen werden
\usepackage{geometry}

\usepackage{makeidx}

% Paket zum Anzeigen der gesamt Seitenzahl '\pageref{lastpage}'
\usepackage{lastpage}



% Paket hyperref für genauere Einstellungen der PDF Erstellung
\usepackage[%
  pdftitle={Das Skorokhod Einbettungs Problem und Hedging von Double
  Barrier Optionen},%                           Titel des PDF Dokuments.
  pdfauthor={Friedel Ziegelmayer},%             Autor des PDF Dokuments.
  pdfsubject={LaTex},%                    	Thema des PDF Dokuments.
  pdfcreator={LaTeX with hyperref},% 		Erzeuger des PDF Dokuments.
  pdfkeywords={},%                        	auch für PDF Dokumente indexiert.
  pdfpagemode=UseOutlines,%                     Inhaltsverzeichnis anzeigen beim Öffnen
  pdfdisplaydoctitle=true,%                     Dokumenttitel statt Dateiname anzeigen.
  pdflang=de%                                   Sprache des Dokuments.
  pdfstartview={FitH}%			        Startanzeigeneinstellung
]{hyperref}

% Pakete für mehr Verweise
\usepackage{
  nameref,      % \nameref
  cleveref,     % \cref   cleveref after! hyperref
}

% Pakte um Counterproblem mit hyperref zu umgehen
\usepackage{aliascnt}
 
%==============================================================================
%				Paketeinbindungen mathematische Pakte
%==============================================================================


% Paket für mehr spaltige Formeln
%\usepackage{mhequ} 

% Standardpaket zur Erweiterung der Umgebungen 'array' und 'tabular' zum 
% Erstellen von tabellenähnlichen Strukturen
\usepackage{array} 

% Standardpaket zur Verbesserung des mathematischen Zeichensatzes
\usepackage{amsmath}

% Paket für den mathematischen Textsatz zur Erweiterung und Verbesserung von 
% amsmath
\usepackage{mathtools}

% Paket zur Darstellung von Theoremstyles 
% Definition & Verwendung wird weiter unten vorgenommen
\usepackage{amsthm}




% Paket zur Darstellung der doppelt gestrichenen Buchstaben 
% (Natürliche Zahlen usw.)
\usepackage{dsfont}

% verschiedene Pakte für zusätzliche Sonderzeichen und mathematische Symbole
\usepackage{amssymb}
\usepackage{stmaryrd}
\usepackage{bbding}
\usepackage{wasysym}

% glossaries wird benutzt umd Indexe zu erstellen
\usepackage[toc]{glossaries}
\makeglossaries




%==============================================================================
%				Einstellungen Schrift und Aussehen
%==============================================================================



% Einstellen verschiedener Großen, wie Textbreite, Texthöhe usw.
% für alle Optionen siehe Paketdokumentation 'geometry'
\geometry{%
	outer=35mm, 
	inner=30mm, 
	top=35mm, 
	bottom=35mm,
	textwidth=145mm, 
	textheight=220mm, 
	marginparsep=3mm, 
	pdftex}

% Definiere die Farbe für Links z.B. im Inhaltsverzeichnis
\definecolor{LinkColor}{rgb}{0,0,0.5}

% weitere Definitionen zum Aussehen und der Verwendung von Links 
\hypersetup{%
	colorlinks=true,%        Aktivieren von farbigen Links im Dokument (keine Rahmen)
	linkcolor=LinkColor,%    Farbe festlegen.
	citecolor=LinkColor,%    Farbe festlegen.
	filecolor=LinkColor,%    Farbe festlegen.
	menucolor=LinkColor,%    Farbe festlegen.
	urlcolor=LinkColor,%     Farbe von URL's im Dokument.
	bookmarksnumbered=true%  Überschriftsnummerierung im PDF Inhalt anzeigen.
}

 
% Definition der Counter für die Abschnittsnummerierung

% I Chapter
\renewcommand \thechapter {\Roman{chapter}}
% 1 Section 
\renewcommand \thesection {\arabic{section}}
% §1 Subsection
\renewcommand \thesubsection {\arabic{section}.\arabic{subsection}}
% §1 Subsection.1 Subsubsection
\renewcommand \thesubsubsection {§\arabic{subsubsection}}


% Definition der Schriften für die Überschriften 
\setkomafont{chapter}{\LARGE\bf}
\setkomafont{section}{\Large\bf}		
\setkomafont{subsection}{\large\bf}

% Setze die Anzahl der Unternummerierungen die möglich sind (z.B. 1.2.1.)
\setcounter{secnumdepth}{3}

\renewcommand{\theequation}{\arabic{equation}}

% Laden der Bibliographie
\usepackage[style=numeric,subentry]{biblatex}
% Lade alle BibTexeinträge unabhängig davon ob sie referenziert werden
% oder nicht
\nocite{*}
\bibliography{bib/cites.bib}



%==============================================================================
%		Einstellungen für Kopf- und Fußzeilen mit fancyhdr
%==============================================================================

% Definiere einen pagestyle 'fancy'
\pagestyle{fancy}
	
	% Setze alles zurück	
	\fancyhf{}
	 
	% Chapter wird nur in den  Dokumentklassen 'book' and 'report' verwendet 
	\renewcommand{\chaptermark}[1]{%
		\markboth{\chaptername \ \thechapter.\ #1%		
		}{}}
	
	\renewcommand{\sectionmark}[1]{%
		\markboth{#1 %
		}{}}
	\renewcommand{\subsectionmark}[1]{% 
		\markright{#1 %
		}{}} 
	
	% Definiere ob und wenn ja wie stark die Linien vor bzw. nach Kopf- 
        % und Fußzeile sind
	\renewcommand{\headrulewidth}{.5pt} 
	\renewcommand{\footrulewidth}{.5pt}
	
	% Einstellen der eigentlichen Information für die Kopfzeile
	\fancyhead[LE,RO]{\footnotesize{}}
	\fancyhead[CO]{\footnotesize{\rightmark}}
        \fancyhead[CE]{\footnotesize{\leftmark}}
	\fancyhead[RE,LO]{\footnotesize{}}
	 
	% Einstellen der eigentlichen Information für die Fußzeile
	\fancyfoot[LE,RO]{\footnotesize{}
	\fancyfoot[CE,CO]{\footnotesize{\thepage\,von \pageref{LastPage}}}}
	\fancyfoot[RE,LO]{\footnotesize{}}







%==============================================================================
%                     Definieren eigener Befehle
%==============================================================================

% Definiere den Befehl '\HRule' um eine 0.3mm dicke Haarline zeichnen zu können
\newcommand{\HRule}{\rule{\linewidth}{0.3mm}}


% Definiere die standardmäßig verwendeten mathematischen Zahlensymbole
%  (Verwendung des Paktes dsfonts) 
\DeclareMathOperator{\N}{\mathds{N}}			% natürliche Zahlen |N
\DeclareMathOperator{\No}{\mathds{N}\setminus\{0\}}	% natürliche Zahlen |N ohne die Null
\DeclareMathOperator{\Z}{\mathds{Z}}			% ganze Zahlen |Z
\DeclareMathOperator{\Q}{\mathds{Q}}			% rationale Zahlen |Q
\DeclareMathOperator{\R}{\mathds{R}}			% reelle Zahlen |R
\DeclareMathOperator{\C}{\mathds{C}}			% komplexe Zahlen |C
\DeclareMathOperator{\Hq}{\mathds{H}}			% Quaterninonen |H
\DeclareMathOperator{\F}{\mathds{F}}			% |F
\DeclareMathOperator{\LL}{\mathds{L}}			% |L
\DeclareMathOperator{\kk}{\mathds{k}}			% |k
\DeclareMathOperator{\PP}{\mathds{P}}			% |P
\DeclareMathOperator{\E}{\mathds{E}}			% |E



% Definition der mathematischen Operatoren für geschwungene Buchstaben

\DeclareMathOperator{\Aa}{\mathcal{A}}
\DeclareMathOperator{\Bb}{\mathcal{B}}
\DeclareMathOperator{\Cc}{\mathcal{C}}
\DeclareMathOperator{\Dd}{\mathcal{D}}
\DeclareMathOperator{\Ee}{\mathcal{E}}
\DeclareMathOperator{\Ff}{\mathcal{F}}
\DeclareMathOperator{\Gg}{\mathcal{G}}
\DeclareMathOperator{\Hh}{\mathcal{H}}
\DeclareMathOperator{\Ii}{\mathcal{I}}
\DeclareMathOperator{\Jj}{\mathcal{J}}
\DeclareMathOperator{\Kk}{\mathcal{K}}
\DeclareMathOperator{\Ll}{\mathcal{L}}
\DeclareMathOperator{\Mm}{\mathcal{M}}
\DeclareMathOperator{\Nn}{\mathcal{N}}
\DeclareMathOperator{\Oo}{\mathcal{O}}
\DeclareMathOperator{\Pp}{\mathcal{P}}
\DeclareMathOperator{\Qq}{\mathcal{Q}}
\DeclareMathOperator{\Rr}{\mathcal{R}}
\DeclareMathOperator{\Ss}{\mathcal{S}}
\DeclareMathOperator{\Tt}{\mathcal{T}}
\DeclareMathOperator{\Uu}{\mathcal{U}}
\DeclareMathOperator{\Vv}{\mathcal{V}}
\DeclareMathOperator{\Ww}{\mathcal{W}}
\DeclareMathOperator{\Xx}{\mathcal{X}}
\DeclareMathOperator{\Yy}{\mathcal{Y}}
\DeclareMathOperator{\Zz}{\mathcal{Z}}

% mathematischer Operator für Indikatorfunktionen mit bold-font
%\DeclareMathOperator{\1}{\mathbf{1}}

% mathematischer Operator für Indikatorfunktionen mit doppel Strich
\DeclareMathOperator{\I1}{\mathds{1}}

% verschiedene andere mathematische Operatoren 
\DeclareMathOperator{\Lp}{L}
\DeclareMathOperator{\Det}{det}
\DeclareMathOperator{\Map}{Map}
\DeclareMathOperator{\Mult}{Mult}
\DeclareMathOperator{\Alt}{Alt}
\DeclareMathOperator{\Mat}{Mat}
\DeclareMathOperator{\sgn}{sign}
\DeclareMathOperator{\Gl}{GL}
\DeclareMathOperator{\rk}{rk}
\DeclareMathOperator{\Bij}{Bij}
\DeclareMathOperator{\leit}{\textit{l}}
\DeclareMathOperator{\RT}{\mathnormal{R[T]}}
\DeclareMathOperator{\ggT}{ggT}
\DeclareMathOperator{\End}{End}
\DeclareMathOperator{\Hom}{Hom}
\DeclareMathOperator{\Invten}{Invten}
\DeclareMathOperator{\Id}{id}
\DeclareMathOperator{\cha}{char}
\DeclareMathOperator{\im}{im}
\DeclareMathOperator{\Tr}{Tr}
\DeclareMathOperator{\Bil}{Bil}
\DeclareMathOperator{\Aut}{Aut}
\DeclareMathOperator{\Orth}{O}
\DeclareMathOperator{\SO}{SO}
\DeclareMathOperator{\SL}{SL}
\DeclareMathOperator{\Sp}{Sp}
\DeclareMathOperator{\sk}{\mathnormal{\langle\, ,\,\rangle}}
\DeclareMathOperator{\SU}{SU}
\DeclareMathOperator{\Un}{U}
\DeclareMathOperator{\supp}{supp}

\DeclareMathOperator{\argmax}{\text{arg max}}
\DeclareMathOperator{\argmin}{\text{arg min}}
\DeclareMathOperator{\arginf}{\text{arg inf}}
\DeclareMathOperator{\argsup}{\text{arg sup}}


%==============================================================================
%				Einstellungen der theoremstyles
%==============================================================================

% Schreibe die Nummerierung vor die Bezeichnung
% 1.2 Satz. statt Satz 2.1. 
%\swapnumbers

% Standard theoremstyle für Sätze, Lemmata, usw.
\newtheoremstyle{plainn}% 	Name
{1.5pt}%      			Abstand zur Vorzeile (leer = default Wert)
{1.5pt}%      			Abstand zur Folgezeile
{\itshape}% 			Schriftart Body
{}%         			Einzug (leer = kein Einzug, \parindent = paragraph Einzug)
{\bfseries}% 			Schriftart Überschrift
{.}%        			Zeichen nach der Überschrift
{.5em}%     			Platz nach der Überschrift (\newline ruft einen Zeilenumbruch hervor)        							
{}% 				Thm head spec

% theoremstyle für Anmerkungen
\newtheoremstyle{anmerkung}% 	Name
{1.5pt}%      			Abstand zur Vorzeile (leer = default Wert)
{1.5pt}%      			Abstand zur Folgezeile
{}% 				Schriftart Body
{}%         			Einzug (leer = kein Einzug, \parindent = paragraph Einzug)
{}%	 			Schriftart Überschrift
{.}%        			Zeichen nach der Überschrift
{}%     			Platz nach der Überschrift (\newline ruft einen Zeilenumbruch hervor)        							
{}% 				Thm head spec


% Zuweisung der theoremstyles zu den einzelnen Umgebungen
% Umgebungen mit * sind unnummeriert


\addto\extrasngerman{%
  \def\satzautorefname{Satz}
  \def\lemmaautorefname{Lemma}  
  \def\theoremautorefname{Theorem}
  \def\korollarautorefname{Korollar}
  \def\propositionautorefname{Proposition}
  \def\beispielname{Beispiel}
  \def\definitionautorefname{Definition}
  \def\bemerkungautorefname{Bemerkung}
  \def\problemautorefname{Problem}
}

\theoremstyle{plainn}
  % verwende die Nummerierung von 'subsection' zusätzlich zur normalen Nummerierung
  \newtheorem{satz}{Satz}[section]
  \newtheorem*{satz*}{Satz}
  
  
  % verwende die selbe Nummerierung wie 'satz'
  \newaliascnt{lemma}{satz}
  \newtheorem{lem}[lemma]{Lemma}
  \newtheorem*{lem*}{Lemma}

  % verwende die selbe Nummerierung wie 'satz'
  \newaliascnt{theorem}{satz}
  \newtheorem{theorem}[theorem]{Theorem}
  \newtheorem*{theorem*}{Theorem}

  % verwende die selbe Nummerierung wie 'satz'
  \newaliascnt{korollar}{satz}
  \newtheorem{kor}[korollar]{Korollar}
  \newtheorem*{kor*}{Korollar}	

  % verwende die selbe Nummerierung wie 'satz'
  \newaliascnt{proposition}{satz}  
  \newtheorem{prop}[proposition]{Proposition}
  \newtheorem*{prop*}{Proposition}

  % verwende die selbe Nummerierung wie 'satz'
  \newaliascnt{problem}{satz}  
  \newtheorem{prob}[problem]{Problem}
  \newtheorem*{prob*}{Problem}


\theoremstyle{definition}
  % verwende die selbe Nummerierung wie 'satz'
  \newaliascnt{beispiel}{satz}
  \newtheorem{bsp}[beispiel]{Beispiel}
  \newtheorem*{bsp*}{Beispiel}   

  % verwende die selbe Nummerierung wie 'satz'
  \newaliascnt{definition}{satz}
  \newtheorem{Def}[definition]{Definition}
  \newtheorem*{Def*}{Definition}

  % verwende die selbe Nummerierung wie 'satz'
  \newaliascnt{bemerkung}{satz}
  \newtheorem{bem}[bemerkung]{Bemerkung}
  \newtheorem*{bem*}{Bemerkung}


\theoremstyle{remark} 
  \newtheorem*{bew*}{Beweis}


% spezielle Umgebung um eine 'leere Umgebung zu haben'
\theoremstyle{anmerkung}
  \newtheorem*{anm}{ }



%==============================================================================
%==============================================================================
%				Ende des Headers
%==============================================================================
%============================================================================== einbinden
% 
%==============================================================================
%==============================================================================



% Definition der 'documentclass' 
% steht immer zu Beginn einer Datei

\documentclass%
[%
%  pdftex,%              PDFTex verwenden da ausschliesslich pdf-files erzeugt werden.
  a4paper,%             Verwendung von A4=Papier
  twoside,%             Zweiseitiger Druck
  11pt,%                Einstellen der Schriftgröße
  halfparskip,%         Halber Zeilenbstand zwischen Absätzen.
  ngerman,%             Deutsche Sprache
  % chapterprefix,%     Kapitel mit 'Kapitel' anschreiben.
  % bibtotocnumbered,%  Literaturverzeichnis im Inhaltsverzeichnis nummeriert einfügen.
  % idxtotoc%           Index ins Inhaltsverzeichnis einfügen.
]{scrreprt}%

%==============================================================================
%		Paketeinbindungen Schrift und Aussehen
%==============================================================================

% fontencoding festlegen (T1 i.A. die einzig sinnvolle Option für LaTex)
% d.h. dies ist die Kodierung für den Output
\usepackage[T1]{fontenc}

% inputencoding festlegen 
% (Standard ist 'latin1', unter Apple kann 'applemac' vorkommen, 'utf8' wird 
% verwendet für möglichst sauberen Austausch
% zwischen Dateisystemen)
% d.h. dies ist die Kodierung der .tex Dateien und kann durchaus varieren
\usepackage[utf8]{inputenc}

% mit diesem Paket werden die Rechschreibregeln für Wortrennung o.ä. festgelegt
% 'ngerman' entspricht der Neuen Deutschen Rechtschreibung, wohingegen 'german'
% für die Alte verwendet wird 
\usepackage[ngerman]{babel} 

% Paket zur Spezifizierung der Anführungszeichen
\usepackage[style=swiss]{csquotes}

% Benutze die Schriftfamilie 'helvet
\usepackage{helvet} 
 
% Paket zum Einstellen der Abstände zwischen den Zeilen
% Optionen sind 'singlespacing' 'onehalfspacing' und 'doublespacing'
\usepackage[onehalfspacing]{setspace} 

% Paket zum Einden von Grafiken '\includegraphics{}' und weiteren graphischen 
% Einstellungen 
\usepackage{graphicx}
\usepackage{subfigure}

 
% Pakte um Farben verwenden zu können 
\usepackage[usenames]{color} 

% Pakete um Grafiken erzeugen zu können
\usepackage{pst-plot}
\usepackage{pstricks}


% Paket zur Quellcodeformatierung in LaTex, für die Verwendung siehe die 
% Paketdokumentation 
\usepackage{listings}

% Paket zur schnellen Änderung von 'enumerate'
% Verwendung: 
% \begin{enumerate}[label=<your label>]
%	\item
%	..
%	\item
%  \end{enumerate}
%
% für <your label> schreibe die gewünschte Nummerierung wie z.B. '(\arabic*)'
% um (1), (2), usw. zu erhalten
\usepackage{enumitem}

% Paket zum erstellen eines Anhangs
\usepackage{appendix}

\usepackage{titlesec}

% Paket zur einfachen Anpassung der Kopf- und Fußzeilen
% Siehe weiter unten zur Verwendung 
\usepackage{fancyhdr}

% Paket zur genauen Defnition der einzelnen Größen im Layout des Dokuments
% siehe weiter unten '\geometry' wo die konkreten Einstellungen vorgenommen werden
\usepackage{geometry}

\usepackage{makeidx}

% Paket zum Anzeigen der gesamt Seitenzahl '\pageref{lastpage}'
\usepackage{lastpage}



% Paket hyperref für genauere Einstellungen der PDF Erstellung
\usepackage[%
  pdftitle={Das Skorokhod Einbettungs Problem und Hedging von Double
  Barrier Optionen},%                           Titel des PDF Dokuments.
  pdfauthor={Friedel Ziegelmayer},%             Autor des PDF Dokuments.
  pdfsubject={LaTex},%                    	Thema des PDF Dokuments.
  pdfcreator={LaTeX with hyperref},% 		Erzeuger des PDF Dokuments.
  pdfkeywords={},%                        	auch für PDF Dokumente indexiert.
  pdfpagemode=UseOutlines,%                     Inhaltsverzeichnis anzeigen beim Öffnen
  pdfdisplaydoctitle=true,%                     Dokumenttitel statt Dateiname anzeigen.
  pdflang=de%                                   Sprache des Dokuments.
  pdfstartview={FitH}%			        Startanzeigeneinstellung
]{hyperref}

% Pakete für mehr Verweise
\usepackage{
  nameref,      % \nameref
  cleveref,     % \cref   cleveref after! hyperref
}

% Pakte um Counterproblem mit hyperref zu umgehen
\usepackage{aliascnt}
 
%==============================================================================
%				Paketeinbindungen mathematische Pakte
%==============================================================================


% Paket für mehr spaltige Formeln
%\usepackage{mhequ} 

% Standardpaket zur Erweiterung der Umgebungen 'array' und 'tabular' zum 
% Erstellen von tabellenähnlichen Strukturen
\usepackage{array} 

% Standardpaket zur Verbesserung des mathematischen Zeichensatzes
\usepackage{amsmath}

% Paket für den mathematischen Textsatz zur Erweiterung und Verbesserung von 
% amsmath
\usepackage{mathtools}

% Paket zur Darstellung von Theoremstyles 
% Definition & Verwendung wird weiter unten vorgenommen
\usepackage{amsthm}




% Paket zur Darstellung der doppelt gestrichenen Buchstaben 
% (Natürliche Zahlen usw.)
\usepackage{dsfont}

% verschiedene Pakte für zusätzliche Sonderzeichen und mathematische Symbole
\usepackage{amssymb}
\usepackage{stmaryrd}
\usepackage{bbding}
\usepackage{wasysym}

% glossaries wird benutzt umd Indexe zu erstellen
\usepackage[toc]{glossaries}
\makeglossaries




%==============================================================================
%				Einstellungen Schrift und Aussehen
%==============================================================================



% Einstellen verschiedener Großen, wie Textbreite, Texthöhe usw.
% für alle Optionen siehe Paketdokumentation 'geometry'
\geometry{%
	outer=35mm, 
	inner=30mm, 
	top=35mm, 
	bottom=35mm,
	textwidth=145mm, 
	textheight=220mm, 
	marginparsep=3mm, 
	pdftex}

% Definiere die Farbe für Links z.B. im Inhaltsverzeichnis
\definecolor{LinkColor}{rgb}{0,0,0.5}

% weitere Definitionen zum Aussehen und der Verwendung von Links 
\hypersetup{%
	colorlinks=true,%        Aktivieren von farbigen Links im Dokument (keine Rahmen)
	linkcolor=LinkColor,%    Farbe festlegen.
	citecolor=LinkColor,%    Farbe festlegen.
	filecolor=LinkColor,%    Farbe festlegen.
	menucolor=LinkColor,%    Farbe festlegen.
	urlcolor=LinkColor,%     Farbe von URL's im Dokument.
	bookmarksnumbered=true%  Überschriftsnummerierung im PDF Inhalt anzeigen.
}

 
% Definition der Counter für die Abschnittsnummerierung

% I Chapter
\renewcommand \thechapter {\Roman{chapter}}
% 1 Section 
\renewcommand \thesection {\arabic{section}}
% §1 Subsection
\renewcommand \thesubsection {\arabic{section}.\arabic{subsection}}
% §1 Subsection.1 Subsubsection
\renewcommand \thesubsubsection {§\arabic{subsubsection}}


% Definition der Schriften für die Überschriften 
\setkomafont{chapter}{\LARGE\bf}
\setkomafont{section}{\Large\bf}		
\setkomafont{subsection}{\large\bf}

% Setze die Anzahl der Unternummerierungen die möglich sind (z.B. 1.2.1.)
\setcounter{secnumdepth}{3}

\renewcommand{\theequation}{\arabic{equation}}

% Laden der Bibliographie
\usepackage[style=numeric,subentry]{biblatex}
% Lade alle BibTexeinträge unabhängig davon ob sie referenziert werden
% oder nicht
\nocite{*}
\bibliography{bib/cites.bib}



%==============================================================================
%		Einstellungen für Kopf- und Fußzeilen mit fancyhdr
%==============================================================================

% Definiere einen pagestyle 'fancy'
\pagestyle{fancy}
	
	% Setze alles zurück	
	\fancyhf{}
	 
	% Chapter wird nur in den  Dokumentklassen 'book' and 'report' verwendet 
	\renewcommand{\chaptermark}[1]{%
		\markboth{\chaptername \ \thechapter.\ #1%		
		}{}}
	
	\renewcommand{\sectionmark}[1]{%
		\markboth{#1 %
		}{}}
	\renewcommand{\subsectionmark}[1]{% 
		\markright{#1 %
		}{}} 
	
	% Definiere ob und wenn ja wie stark die Linien vor bzw. nach Kopf- 
        % und Fußzeile sind
	\renewcommand{\headrulewidth}{.5pt} 
	\renewcommand{\footrulewidth}{.5pt}
	
	% Einstellen der eigentlichen Information für die Kopfzeile
	\fancyhead[LE,RO]{\footnotesize{}}
	\fancyhead[CO]{\footnotesize{\rightmark}}
        \fancyhead[CE]{\footnotesize{\leftmark}}
	\fancyhead[RE,LO]{\footnotesize{}}
	 
	% Einstellen der eigentlichen Information für die Fußzeile
	\fancyfoot[LE,RO]{\footnotesize{}
	\fancyfoot[CE,CO]{\footnotesize{\thepage\,von \pageref{LastPage}}}}
	\fancyfoot[RE,LO]{\footnotesize{}}







%==============================================================================
%                     Definieren eigener Befehle
%==============================================================================

% Definiere den Befehl '\HRule' um eine 0.3mm dicke Haarline zeichnen zu können
\newcommand{\HRule}{\rule{\linewidth}{0.3mm}}


% Definiere die standardmäßig verwendeten mathematischen Zahlensymbole
%  (Verwendung des Paktes dsfonts) 
\DeclareMathOperator{\N}{\mathds{N}}			% natürliche Zahlen |N
\DeclareMathOperator{\No}{\mathds{N}\setminus\{0\}}	% natürliche Zahlen |N ohne die Null
\DeclareMathOperator{\Z}{\mathds{Z}}			% ganze Zahlen |Z
\DeclareMathOperator{\Q}{\mathds{Q}}			% rationale Zahlen |Q
\DeclareMathOperator{\R}{\mathds{R}}			% reelle Zahlen |R
\DeclareMathOperator{\C}{\mathds{C}}			% komplexe Zahlen |C
\DeclareMathOperator{\Hq}{\mathds{H}}			% Quaterninonen |H
\DeclareMathOperator{\F}{\mathds{F}}			% |F
\DeclareMathOperator{\LL}{\mathds{L}}			% |L
\DeclareMathOperator{\kk}{\mathds{k}}			% |k
\DeclareMathOperator{\PP}{\mathds{P}}			% |P
\DeclareMathOperator{\E}{\mathds{E}}			% |E



% Definition der mathematischen Operatoren für geschwungene Buchstaben

\DeclareMathOperator{\Aa}{\mathcal{A}}
\DeclareMathOperator{\Bb}{\mathcal{B}}
\DeclareMathOperator{\Cc}{\mathcal{C}}
\DeclareMathOperator{\Dd}{\mathcal{D}}
\DeclareMathOperator{\Ee}{\mathcal{E}}
\DeclareMathOperator{\Ff}{\mathcal{F}}
\DeclareMathOperator{\Gg}{\mathcal{G}}
\DeclareMathOperator{\Hh}{\mathcal{H}}
\DeclareMathOperator{\Ii}{\mathcal{I}}
\DeclareMathOperator{\Jj}{\mathcal{J}}
\DeclareMathOperator{\Kk}{\mathcal{K}}
\DeclareMathOperator{\Ll}{\mathcal{L}}
\DeclareMathOperator{\Mm}{\mathcal{M}}
\DeclareMathOperator{\Nn}{\mathcal{N}}
\DeclareMathOperator{\Oo}{\mathcal{O}}
\DeclareMathOperator{\Pp}{\mathcal{P}}
\DeclareMathOperator{\Qq}{\mathcal{Q}}
\DeclareMathOperator{\Rr}{\mathcal{R}}
\DeclareMathOperator{\Ss}{\mathcal{S}}
\DeclareMathOperator{\Tt}{\mathcal{T}}
\DeclareMathOperator{\Uu}{\mathcal{U}}
\DeclareMathOperator{\Vv}{\mathcal{V}}
\DeclareMathOperator{\Ww}{\mathcal{W}}
\DeclareMathOperator{\Xx}{\mathcal{X}}
\DeclareMathOperator{\Yy}{\mathcal{Y}}
\DeclareMathOperator{\Zz}{\mathcal{Z}}

% mathematischer Operator für Indikatorfunktionen mit bold-font
%\DeclareMathOperator{\1}{\mathbf{1}}

% mathematischer Operator für Indikatorfunktionen mit doppel Strich
\DeclareMathOperator{\I1}{\mathds{1}}

% verschiedene andere mathematische Operatoren 
\DeclareMathOperator{\Lp}{L}
\DeclareMathOperator{\Det}{det}
\DeclareMathOperator{\Map}{Map}
\DeclareMathOperator{\Mult}{Mult}
\DeclareMathOperator{\Alt}{Alt}
\DeclareMathOperator{\Mat}{Mat}
\DeclareMathOperator{\sgn}{sign}
\DeclareMathOperator{\Gl}{GL}
\DeclareMathOperator{\rk}{rk}
\DeclareMathOperator{\Bij}{Bij}
\DeclareMathOperator{\leit}{\textit{l}}
\DeclareMathOperator{\RT}{\mathnormal{R[T]}}
\DeclareMathOperator{\ggT}{ggT}
\DeclareMathOperator{\End}{End}
\DeclareMathOperator{\Hom}{Hom}
\DeclareMathOperator{\Invten}{Invten}
\DeclareMathOperator{\Id}{id}
\DeclareMathOperator{\cha}{char}
\DeclareMathOperator{\im}{im}
\DeclareMathOperator{\Tr}{Tr}
\DeclareMathOperator{\Bil}{Bil}
\DeclareMathOperator{\Aut}{Aut}
\DeclareMathOperator{\Orth}{O}
\DeclareMathOperator{\SO}{SO}
\DeclareMathOperator{\SL}{SL}
\DeclareMathOperator{\Sp}{Sp}
\DeclareMathOperator{\sk}{\mathnormal{\langle\, ,\,\rangle}}
\DeclareMathOperator{\SU}{SU}
\DeclareMathOperator{\Un}{U}
\DeclareMathOperator{\supp}{supp}

\DeclareMathOperator{\argmax}{\text{arg max}}
\DeclareMathOperator{\argmin}{\text{arg min}}
\DeclareMathOperator{\arginf}{\text{arg inf}}
\DeclareMathOperator{\argsup}{\text{arg sup}}


%==============================================================================
%				Einstellungen der theoremstyles
%==============================================================================

% Schreibe die Nummerierung vor die Bezeichnung
% 1.2 Satz. statt Satz 2.1. 
%\swapnumbers

% Standard theoremstyle für Sätze, Lemmata, usw.
\newtheoremstyle{plainn}% 	Name
{1.5pt}%      			Abstand zur Vorzeile (leer = default Wert)
{1.5pt}%      			Abstand zur Folgezeile
{\itshape}% 			Schriftart Body
{}%         			Einzug (leer = kein Einzug, \parindent = paragraph Einzug)
{\bfseries}% 			Schriftart Überschrift
{.}%        			Zeichen nach der Überschrift
{.5em}%     			Platz nach der Überschrift (\newline ruft einen Zeilenumbruch hervor)        							
{}% 				Thm head spec

% theoremstyle für Anmerkungen
\newtheoremstyle{anmerkung}% 	Name
{1.5pt}%      			Abstand zur Vorzeile (leer = default Wert)
{1.5pt}%      			Abstand zur Folgezeile
{}% 				Schriftart Body
{}%         			Einzug (leer = kein Einzug, \parindent = paragraph Einzug)
{}%	 			Schriftart Überschrift
{.}%        			Zeichen nach der Überschrift
{}%     			Platz nach der Überschrift (\newline ruft einen Zeilenumbruch hervor)        							
{}% 				Thm head spec


% Zuweisung der theoremstyles zu den einzelnen Umgebungen
% Umgebungen mit * sind unnummeriert


\addto\extrasngerman{%
  \def\satzautorefname{Satz}
  \def\lemmaautorefname{Lemma}  
  \def\theoremautorefname{Theorem}
  \def\korollarautorefname{Korollar}
  \def\propositionautorefname{Proposition}
  \def\beispielname{Beispiel}
  \def\definitionautorefname{Definition}
  \def\bemerkungautorefname{Bemerkung}
  \def\problemautorefname{Problem}
}

\theoremstyle{plainn}
  % verwende die Nummerierung von 'subsection' zusätzlich zur normalen Nummerierung
  \newtheorem{satz}{Satz}[section]
  \newtheorem*{satz*}{Satz}
  
  
  % verwende die selbe Nummerierung wie 'satz'
  \newaliascnt{lemma}{satz}
  \newtheorem{lem}[lemma]{Lemma}
  \newtheorem*{lem*}{Lemma}

  % verwende die selbe Nummerierung wie 'satz'
  \newaliascnt{theorem}{satz}
  \newtheorem{theorem}[theorem]{Theorem}
  \newtheorem*{theorem*}{Theorem}

  % verwende die selbe Nummerierung wie 'satz'
  \newaliascnt{korollar}{satz}
  \newtheorem{kor}[korollar]{Korollar}
  \newtheorem*{kor*}{Korollar}	

  % verwende die selbe Nummerierung wie 'satz'
  \newaliascnt{proposition}{satz}  
  \newtheorem{prop}[proposition]{Proposition}
  \newtheorem*{prop*}{Proposition}

  % verwende die selbe Nummerierung wie 'satz'
  \newaliascnt{problem}{satz}  
  \newtheorem{prob}[problem]{Problem}
  \newtheorem*{prob*}{Problem}


\theoremstyle{definition}
  % verwende die selbe Nummerierung wie 'satz'
  \newaliascnt{beispiel}{satz}
  \newtheorem{bsp}[beispiel]{Beispiel}
  \newtheorem*{bsp*}{Beispiel}   

  % verwende die selbe Nummerierung wie 'satz'
  \newaliascnt{definition}{satz}
  \newtheorem{Def}[definition]{Definition}
  \newtheorem*{Def*}{Definition}

  % verwende die selbe Nummerierung wie 'satz'
  \newaliascnt{bemerkung}{satz}
  \newtheorem{bem}[bemerkung]{Bemerkung}
  \newtheorem*{bem*}{Bemerkung}


\theoremstyle{remark} 
  \newtheorem*{bew*}{Beweis}


% spezielle Umgebung um eine 'leere Umgebung zu haben'
\theoremstyle{anmerkung}
  \newtheorem*{anm}{ }



%==============================================================================
%==============================================================================
%				Ende des Headers
%==============================================================================
%============================================================================== einbinden
% 
%==============================================================================
%==============================================================================



% Definition der 'documentclass' 
% steht immer zu Beginn einer Datei

\documentclass%
[%
%  pdftex,%              PDFTex verwenden da ausschliesslich pdf-files erzeugt werden.
  a4paper,%             Verwendung von A4=Papier
  twoside,%             Zweiseitiger Druck
  11pt,%                Einstellen der Schriftgröße
  halfparskip,%         Halber Zeilenbstand zwischen Absätzen.
  ngerman,%             Deutsche Sprache
  % chapterprefix,%     Kapitel mit 'Kapitel' anschreiben.
  % bibtotocnumbered,%  Literaturverzeichnis im Inhaltsverzeichnis nummeriert einfügen.
  % idxtotoc%           Index ins Inhaltsverzeichnis einfügen.
]{scrreprt}%

%==============================================================================
%		Paketeinbindungen Schrift und Aussehen
%==============================================================================

% fontencoding festlegen (T1 i.A. die einzig sinnvolle Option für LaTex)
% d.h. dies ist die Kodierung für den Output
\usepackage[T1]{fontenc}

% inputencoding festlegen 
% (Standard ist 'latin1', unter Apple kann 'applemac' vorkommen, 'utf8' wird 
% verwendet für möglichst sauberen Austausch
% zwischen Dateisystemen)
% d.h. dies ist die Kodierung der .tex Dateien und kann durchaus varieren
\usepackage[utf8]{inputenc}

% mit diesem Paket werden die Rechschreibregeln für Wortrennung o.ä. festgelegt
% 'ngerman' entspricht der Neuen Deutschen Rechtschreibung, wohingegen 'german'
% für die Alte verwendet wird 
\usepackage[ngerman]{babel} 

% Paket zur Spezifizierung der Anführungszeichen
\usepackage[style=swiss]{csquotes}

% Benutze die Schriftfamilie 'helvet
\usepackage{helvet} 
 
% Paket zum Einstellen der Abstände zwischen den Zeilen
% Optionen sind 'singlespacing' 'onehalfspacing' und 'doublespacing'
\usepackage[onehalfspacing]{setspace} 

% Paket zum Einden von Grafiken '\includegraphics{}' und weiteren graphischen 
% Einstellungen 
\usepackage{graphicx}
\usepackage{subfigure}

 
% Pakte um Farben verwenden zu können 
\usepackage[usenames]{color} 

% Pakete um Grafiken erzeugen zu können
\usepackage{pst-plot}
\usepackage{pstricks}


% Paket zur Quellcodeformatierung in LaTex, für die Verwendung siehe die 
% Paketdokumentation 
\usepackage{listings}

% Paket zur schnellen Änderung von 'enumerate'
% Verwendung: 
% \begin{enumerate}[label=<your label>]
%	\item
%	..
%	\item
%  \end{enumerate}
%
% für <your label> schreibe die gewünschte Nummerierung wie z.B. '(\arabic*)'
% um (1), (2), usw. zu erhalten
\usepackage{enumitem}

% Paket zum erstellen eines Anhangs
\usepackage{appendix}

\usepackage{titlesec}

% Paket zur einfachen Anpassung der Kopf- und Fußzeilen
% Siehe weiter unten zur Verwendung 
\usepackage{fancyhdr}

% Paket zur genauen Defnition der einzelnen Größen im Layout des Dokuments
% siehe weiter unten '\geometry' wo die konkreten Einstellungen vorgenommen werden
\usepackage{geometry}

\usepackage{makeidx}

% Paket zum Anzeigen der gesamt Seitenzahl '\pageref{lastpage}'
\usepackage{lastpage}



% Paket hyperref für genauere Einstellungen der PDF Erstellung
\usepackage[%
  pdftitle={Das Skorokhod Einbettungs Problem und Hedging von Double
  Barrier Optionen},%                           Titel des PDF Dokuments.
  pdfauthor={Friedel Ziegelmayer},%             Autor des PDF Dokuments.
  pdfsubject={LaTex},%                    	Thema des PDF Dokuments.
  pdfcreator={LaTeX with hyperref},% 		Erzeuger des PDF Dokuments.
  pdfkeywords={},%                        	auch für PDF Dokumente indexiert.
  pdfpagemode=UseOutlines,%                     Inhaltsverzeichnis anzeigen beim Öffnen
  pdfdisplaydoctitle=true,%                     Dokumenttitel statt Dateiname anzeigen.
  pdflang=de%                                   Sprache des Dokuments.
  pdfstartview={FitH}%			        Startanzeigeneinstellung
]{hyperref}

% Pakete für mehr Verweise
\usepackage{
  nameref,      % \nameref
  cleveref,     % \cref   cleveref after! hyperref
}

% Pakte um Counterproblem mit hyperref zu umgehen
\usepackage{aliascnt}
 
%==============================================================================
%				Paketeinbindungen mathematische Pakte
%==============================================================================


% Paket für mehr spaltige Formeln
%\usepackage{mhequ} 

% Standardpaket zur Erweiterung der Umgebungen 'array' und 'tabular' zum 
% Erstellen von tabellenähnlichen Strukturen
\usepackage{array} 

% Standardpaket zur Verbesserung des mathematischen Zeichensatzes
\usepackage{amsmath}

% Paket für den mathematischen Textsatz zur Erweiterung und Verbesserung von 
% amsmath
\usepackage{mathtools}

% Paket zur Darstellung von Theoremstyles 
% Definition & Verwendung wird weiter unten vorgenommen
\usepackage{amsthm}




% Paket zur Darstellung der doppelt gestrichenen Buchstaben 
% (Natürliche Zahlen usw.)
\usepackage{dsfont}

% verschiedene Pakte für zusätzliche Sonderzeichen und mathematische Symbole
\usepackage{amssymb}
\usepackage{stmaryrd}
\usepackage{bbding}
\usepackage{wasysym}

% glossaries wird benutzt umd Indexe zu erstellen
\usepackage[toc]{glossaries}
\makeglossaries




%==============================================================================
%				Einstellungen Schrift und Aussehen
%==============================================================================



% Einstellen verschiedener Großen, wie Textbreite, Texthöhe usw.
% für alle Optionen siehe Paketdokumentation 'geometry'
\geometry{%
	outer=35mm, 
	inner=30mm, 
	top=35mm, 
	bottom=35mm,
	textwidth=145mm, 
	textheight=220mm, 
	marginparsep=3mm, 
	pdftex}

% Definiere die Farbe für Links z.B. im Inhaltsverzeichnis
\definecolor{LinkColor}{rgb}{0,0,0.5}

% weitere Definitionen zum Aussehen und der Verwendung von Links 
\hypersetup{%
	colorlinks=true,%        Aktivieren von farbigen Links im Dokument (keine Rahmen)
	linkcolor=LinkColor,%    Farbe festlegen.
	citecolor=LinkColor,%    Farbe festlegen.
	filecolor=LinkColor,%    Farbe festlegen.
	menucolor=LinkColor,%    Farbe festlegen.
	urlcolor=LinkColor,%     Farbe von URL's im Dokument.
	bookmarksnumbered=true%  Überschriftsnummerierung im PDF Inhalt anzeigen.
}

 
% Definition der Counter für die Abschnittsnummerierung

% I Chapter
\renewcommand \thechapter {\Roman{chapter}}
% 1 Section 
\renewcommand \thesection {\arabic{section}}
% §1 Subsection
\renewcommand \thesubsection {\arabic{section}.\arabic{subsection}}
% §1 Subsection.1 Subsubsection
\renewcommand \thesubsubsection {§\arabic{subsubsection}}


% Definition der Schriften für die Überschriften 
\setkomafont{chapter}{\LARGE\bf}
\setkomafont{section}{\Large\bf}		
\setkomafont{subsection}{\large\bf}

% Setze die Anzahl der Unternummerierungen die möglich sind (z.B. 1.2.1.)
\setcounter{secnumdepth}{3}

\renewcommand{\theequation}{\arabic{equation}}

% Laden der Bibliographie
\usepackage[style=numeric,subentry]{biblatex}
% Lade alle BibTexeinträge unabhängig davon ob sie referenziert werden
% oder nicht
\nocite{*}
\bibliography{bib/cites.bib}



%==============================================================================
%		Einstellungen für Kopf- und Fußzeilen mit fancyhdr
%==============================================================================

% Definiere einen pagestyle 'fancy'
\pagestyle{fancy}
	
	% Setze alles zurück	
	\fancyhf{}
	 
	% Chapter wird nur in den  Dokumentklassen 'book' and 'report' verwendet 
	\renewcommand{\chaptermark}[1]{%
		\markboth{\chaptername \ \thechapter.\ #1%		
		}{}}
	
	\renewcommand{\sectionmark}[1]{%
		\markboth{#1 %
		}{}}
	\renewcommand{\subsectionmark}[1]{% 
		\markright{#1 %
		}{}} 
	
	% Definiere ob und wenn ja wie stark die Linien vor bzw. nach Kopf- 
        % und Fußzeile sind
	\renewcommand{\headrulewidth}{.5pt} 
	\renewcommand{\footrulewidth}{.5pt}
	
	% Einstellen der eigentlichen Information für die Kopfzeile
	\fancyhead[LE,RO]{\footnotesize{}}
	\fancyhead[CO]{\footnotesize{\rightmark}}
        \fancyhead[CE]{\footnotesize{\leftmark}}
	\fancyhead[RE,LO]{\footnotesize{}}
	 
	% Einstellen der eigentlichen Information für die Fußzeile
	\fancyfoot[LE,RO]{\footnotesize{}
	\fancyfoot[CE,CO]{\footnotesize{\thepage\,von \pageref{LastPage}}}}
	\fancyfoot[RE,LO]{\footnotesize{}}







%==============================================================================
%                     Definieren eigener Befehle
%==============================================================================

% Definiere den Befehl '\HRule' um eine 0.3mm dicke Haarline zeichnen zu können
\newcommand{\HRule}{\rule{\linewidth}{0.3mm}}


% Definiere die standardmäßig verwendeten mathematischen Zahlensymbole
%  (Verwendung des Paktes dsfonts) 
\DeclareMathOperator{\N}{\mathds{N}}			% natürliche Zahlen |N
\DeclareMathOperator{\No}{\mathds{N}\setminus\{0\}}	% natürliche Zahlen |N ohne die Null
\DeclareMathOperator{\Z}{\mathds{Z}}			% ganze Zahlen |Z
\DeclareMathOperator{\Q}{\mathds{Q}}			% rationale Zahlen |Q
\DeclareMathOperator{\R}{\mathds{R}}			% reelle Zahlen |R
\DeclareMathOperator{\C}{\mathds{C}}			% komplexe Zahlen |C
\DeclareMathOperator{\Hq}{\mathds{H}}			% Quaterninonen |H
\DeclareMathOperator{\F}{\mathds{F}}			% |F
\DeclareMathOperator{\LL}{\mathds{L}}			% |L
\DeclareMathOperator{\kk}{\mathds{k}}			% |k
\DeclareMathOperator{\PP}{\mathds{P}}			% |P
\DeclareMathOperator{\E}{\mathds{E}}			% |E



% Definition der mathematischen Operatoren für geschwungene Buchstaben

\DeclareMathOperator{\Aa}{\mathcal{A}}
\DeclareMathOperator{\Bb}{\mathcal{B}}
\DeclareMathOperator{\Cc}{\mathcal{C}}
\DeclareMathOperator{\Dd}{\mathcal{D}}
\DeclareMathOperator{\Ee}{\mathcal{E}}
\DeclareMathOperator{\Ff}{\mathcal{F}}
\DeclareMathOperator{\Gg}{\mathcal{G}}
\DeclareMathOperator{\Hh}{\mathcal{H}}
\DeclareMathOperator{\Ii}{\mathcal{I}}
\DeclareMathOperator{\Jj}{\mathcal{J}}
\DeclareMathOperator{\Kk}{\mathcal{K}}
\DeclareMathOperator{\Ll}{\mathcal{L}}
\DeclareMathOperator{\Mm}{\mathcal{M}}
\DeclareMathOperator{\Nn}{\mathcal{N}}
\DeclareMathOperator{\Oo}{\mathcal{O}}
\DeclareMathOperator{\Pp}{\mathcal{P}}
\DeclareMathOperator{\Qq}{\mathcal{Q}}
\DeclareMathOperator{\Rr}{\mathcal{R}}
\DeclareMathOperator{\Ss}{\mathcal{S}}
\DeclareMathOperator{\Tt}{\mathcal{T}}
\DeclareMathOperator{\Uu}{\mathcal{U}}
\DeclareMathOperator{\Vv}{\mathcal{V}}
\DeclareMathOperator{\Ww}{\mathcal{W}}
\DeclareMathOperator{\Xx}{\mathcal{X}}
\DeclareMathOperator{\Yy}{\mathcal{Y}}
\DeclareMathOperator{\Zz}{\mathcal{Z}}

% mathematischer Operator für Indikatorfunktionen mit bold-font
%\DeclareMathOperator{\1}{\mathbf{1}}

% mathematischer Operator für Indikatorfunktionen mit doppel Strich
\DeclareMathOperator{\I1}{\mathds{1}}

% verschiedene andere mathematische Operatoren 
\DeclareMathOperator{\Lp}{L}
\DeclareMathOperator{\Det}{det}
\DeclareMathOperator{\Map}{Map}
\DeclareMathOperator{\Mult}{Mult}
\DeclareMathOperator{\Alt}{Alt}
\DeclareMathOperator{\Mat}{Mat}
\DeclareMathOperator{\sgn}{sign}
\DeclareMathOperator{\Gl}{GL}
\DeclareMathOperator{\rk}{rk}
\DeclareMathOperator{\Bij}{Bij}
\DeclareMathOperator{\leit}{\textit{l}}
\DeclareMathOperator{\RT}{\mathnormal{R[T]}}
\DeclareMathOperator{\ggT}{ggT}
\DeclareMathOperator{\End}{End}
\DeclareMathOperator{\Hom}{Hom}
\DeclareMathOperator{\Invten}{Invten}
\DeclareMathOperator{\Id}{id}
\DeclareMathOperator{\cha}{char}
\DeclareMathOperator{\im}{im}
\DeclareMathOperator{\Tr}{Tr}
\DeclareMathOperator{\Bil}{Bil}
\DeclareMathOperator{\Aut}{Aut}
\DeclareMathOperator{\Orth}{O}
\DeclareMathOperator{\SO}{SO}
\DeclareMathOperator{\SL}{SL}
\DeclareMathOperator{\Sp}{Sp}
\DeclareMathOperator{\sk}{\mathnormal{\langle\, ,\,\rangle}}
\DeclareMathOperator{\SU}{SU}
\DeclareMathOperator{\Un}{U}
\DeclareMathOperator{\supp}{supp}

\DeclareMathOperator{\argmax}{\text{arg max}}
\DeclareMathOperator{\argmin}{\text{arg min}}
\DeclareMathOperator{\arginf}{\text{arg inf}}
\DeclareMathOperator{\argsup}{\text{arg sup}}


%==============================================================================
%				Einstellungen der theoremstyles
%==============================================================================

% Schreibe die Nummerierung vor die Bezeichnung
% 1.2 Satz. statt Satz 2.1. 
%\swapnumbers

% Standard theoremstyle für Sätze, Lemmata, usw.
\newtheoremstyle{plainn}% 	Name
{1.5pt}%      			Abstand zur Vorzeile (leer = default Wert)
{1.5pt}%      			Abstand zur Folgezeile
{\itshape}% 			Schriftart Body
{}%         			Einzug (leer = kein Einzug, \parindent = paragraph Einzug)
{\bfseries}% 			Schriftart Überschrift
{.}%        			Zeichen nach der Überschrift
{.5em}%     			Platz nach der Überschrift (\newline ruft einen Zeilenumbruch hervor)        							
{}% 				Thm head spec

% theoremstyle für Anmerkungen
\newtheoremstyle{anmerkung}% 	Name
{1.5pt}%      			Abstand zur Vorzeile (leer = default Wert)
{1.5pt}%      			Abstand zur Folgezeile
{}% 				Schriftart Body
{}%         			Einzug (leer = kein Einzug, \parindent = paragraph Einzug)
{}%	 			Schriftart Überschrift
{.}%        			Zeichen nach der Überschrift
{}%     			Platz nach der Überschrift (\newline ruft einen Zeilenumbruch hervor)        							
{}% 				Thm head spec


% Zuweisung der theoremstyles zu den einzelnen Umgebungen
% Umgebungen mit * sind unnummeriert


\addto\extrasngerman{%
  \def\satzautorefname{Satz}
  \def\lemmaautorefname{Lemma}  
  \def\theoremautorefname{Theorem}
  \def\korollarautorefname{Korollar}
  \def\propositionautorefname{Proposition}
  \def\beispielname{Beispiel}
  \def\definitionautorefname{Definition}
  \def\bemerkungautorefname{Bemerkung}
  \def\problemautorefname{Problem}
}

\theoremstyle{plainn}
  % verwende die Nummerierung von 'subsection' zusätzlich zur normalen Nummerierung
  \newtheorem{satz}{Satz}[section]
  \newtheorem*{satz*}{Satz}
  
  
  % verwende die selbe Nummerierung wie 'satz'
  \newaliascnt{lemma}{satz}
  \newtheorem{lem}[lemma]{Lemma}
  \newtheorem*{lem*}{Lemma}

  % verwende die selbe Nummerierung wie 'satz'
  \newaliascnt{theorem}{satz}
  \newtheorem{theorem}[theorem]{Theorem}
  \newtheorem*{theorem*}{Theorem}

  % verwende die selbe Nummerierung wie 'satz'
  \newaliascnt{korollar}{satz}
  \newtheorem{kor}[korollar]{Korollar}
  \newtheorem*{kor*}{Korollar}	

  % verwende die selbe Nummerierung wie 'satz'
  \newaliascnt{proposition}{satz}  
  \newtheorem{prop}[proposition]{Proposition}
  \newtheorem*{prop*}{Proposition}

  % verwende die selbe Nummerierung wie 'satz'
  \newaliascnt{problem}{satz}  
  \newtheorem{prob}[problem]{Problem}
  \newtheorem*{prob*}{Problem}


\theoremstyle{definition}
  % verwende die selbe Nummerierung wie 'satz'
  \newaliascnt{beispiel}{satz}
  \newtheorem{bsp}[beispiel]{Beispiel}
  \newtheorem*{bsp*}{Beispiel}   

  % verwende die selbe Nummerierung wie 'satz'
  \newaliascnt{definition}{satz}
  \newtheorem{Def}[definition]{Definition}
  \newtheorem*{Def*}{Definition}

  % verwende die selbe Nummerierung wie 'satz'
  \newaliascnt{bemerkung}{satz}
  \newtheorem{bem}[bemerkung]{Bemerkung}
  \newtheorem*{bem*}{Bemerkung}


\theoremstyle{remark} 
  \newtheorem*{bew*}{Beweis}


% spezielle Umgebung um eine 'leere Umgebung zu haben'
\theoremstyle{anmerkung}
  \newtheorem*{anm}{ }



%==============================================================================
%==============================================================================
%				Ende des Headers
%==============================================================================
%============================================================================== einbinden
% 
%==============================================================================
%==============================================================================



% Definition der 'documentclass' 
% steht immer zu Beginn einer Datei

\documentclass%
[%
%  pdftex,%              PDFTex verwenden da ausschliesslich pdf-files erzeugt werden.
  a4paper,%             Verwendung von A4=Papier
  twoside,%             Zweiseitiger Druck
  11pt,%                Einstellen der Schriftgröße
  halfparskip,%         Halber Zeilenbstand zwischen Absätzen.
  ngerman,%             Deutsche Sprache
  % chapterprefix,%     Kapitel mit 'Kapitel' anschreiben.
  % bibtotocnumbered,%  Literaturverzeichnis im Inhaltsverzeichnis nummeriert einfügen.
  % idxtotoc%           Index ins Inhaltsverzeichnis einfügen.
]{scrreprt}%

%==============================================================================
%		Paketeinbindungen Schrift und Aussehen
%==============================================================================

% fontencoding festlegen (T1 i.A. die einzig sinnvolle Option für LaTex)
% d.h. dies ist die Kodierung für den Output
\usepackage[T1]{fontenc}

% inputencoding festlegen 
% (Standard ist 'latin1', unter Apple kann 'applemac' vorkommen, 'utf8' wird 
% verwendet für möglichst sauberen Austausch
% zwischen Dateisystemen)
% d.h. dies ist die Kodierung der .tex Dateien und kann durchaus varieren
\usepackage[utf8]{inputenc}

% mit diesem Paket werden die Rechschreibregeln für Wortrennung o.ä. festgelegt
% 'ngerman' entspricht der Neuen Deutschen Rechtschreibung, wohingegen 'german'
% für die Alte verwendet wird 
\usepackage[ngerman]{babel} 

% Paket zur Spezifizierung der Anführungszeichen
\usepackage[style=swiss]{csquotes}

% Benutze die Schriftfamilie 'helvet
\usepackage{helvet} 
 
% Paket zum Einstellen der Abstände zwischen den Zeilen
% Optionen sind 'singlespacing' 'onehalfspacing' und 'doublespacing'
\usepackage[onehalfspacing]{setspace} 

% Paket zum Einden von Grafiken '\includegraphics{}' und weiteren graphischen 
% Einstellungen 
\usepackage{graphicx}
\usepackage{subfigure}

 
% Pakte um Farben verwenden zu können 
\usepackage[usenames]{color} 

% Pakete um Grafiken erzeugen zu können
\usepackage{pst-plot}
\usepackage{pstricks}


% Paket zur Quellcodeformatierung in LaTex, für die Verwendung siehe die 
% Paketdokumentation 
\usepackage{listings}

% Paket zur schnellen Änderung von 'enumerate'
% Verwendung: 
% \begin{enumerate}[label=<your label>]
%	\item
%	..
%	\item
%  \end{enumerate}
%
% für <your label> schreibe die gewünschte Nummerierung wie z.B. '(\arabic*)'
% um (1), (2), usw. zu erhalten
\usepackage{enumitem}

% Paket zum erstellen eines Anhangs
\usepackage{appendix}

\usepackage{titlesec}

% Paket zur einfachen Anpassung der Kopf- und Fußzeilen
% Siehe weiter unten zur Verwendung 
\usepackage{fancyhdr}

% Paket zur genauen Defnition der einzelnen Größen im Layout des Dokuments
% siehe weiter unten '\geometry' wo die konkreten Einstellungen vorgenommen werden
\usepackage{geometry}

\usepackage{makeidx}

% Paket zum Anzeigen der gesamt Seitenzahl '\pageref{lastpage}'
\usepackage{lastpage}



% Paket hyperref für genauere Einstellungen der PDF Erstellung
\usepackage[%
  pdftitle={Das Skorokhod Einbettungs Problem und Hedging von Double
  Barrier Optionen},%                           Titel des PDF Dokuments.
  pdfauthor={Friedel Ziegelmayer},%             Autor des PDF Dokuments.
  pdfsubject={LaTex},%                    	Thema des PDF Dokuments.
  pdfcreator={LaTeX with hyperref},% 		Erzeuger des PDF Dokuments.
  pdfkeywords={},%                        	auch für PDF Dokumente indexiert.
  pdfpagemode=UseOutlines,%                     Inhaltsverzeichnis anzeigen beim Öffnen
  pdfdisplaydoctitle=true,%                     Dokumenttitel statt Dateiname anzeigen.
  pdflang=de%                                   Sprache des Dokuments.
  pdfstartview={FitH}%			        Startanzeigeneinstellung
]{hyperref}

% Pakete für mehr Verweise
\usepackage{
  nameref,      % \nameref
  cleveref,     % \cref   cleveref after! hyperref
}

% Pakte um Counterproblem mit hyperref zu umgehen
\usepackage{aliascnt}
 
%==============================================================================
%				Paketeinbindungen mathematische Pakte
%==============================================================================


% Paket für mehr spaltige Formeln
%\usepackage{mhequ} 

% Standardpaket zur Erweiterung der Umgebungen 'array' und 'tabular' zum 
% Erstellen von tabellenähnlichen Strukturen
\usepackage{array} 

% Standardpaket zur Verbesserung des mathematischen Zeichensatzes
\usepackage{amsmath}

% Paket für den mathematischen Textsatz zur Erweiterung und Verbesserung von 
% amsmath
\usepackage{mathtools}

% Paket zur Darstellung von Theoremstyles 
% Definition & Verwendung wird weiter unten vorgenommen
\usepackage{amsthm}




% Paket zur Darstellung der doppelt gestrichenen Buchstaben 
% (Natürliche Zahlen usw.)
\usepackage{dsfont}

% verschiedene Pakte für zusätzliche Sonderzeichen und mathematische Symbole
\usepackage{amssymb}
\usepackage{stmaryrd}
\usepackage{bbding}
\usepackage{wasysym}

% glossaries wird benutzt umd Indexe zu erstellen
\usepackage[toc]{glossaries}
\makeglossaries




%==============================================================================
%				Einstellungen Schrift und Aussehen
%==============================================================================



% Einstellen verschiedener Großen, wie Textbreite, Texthöhe usw.
% für alle Optionen siehe Paketdokumentation 'geometry'
\geometry{%
	outer=35mm, 
	inner=30mm, 
	top=35mm, 
	bottom=35mm,
	textwidth=145mm, 
	textheight=220mm, 
	marginparsep=3mm, 
	pdftex}

% Definiere die Farbe für Links z.B. im Inhaltsverzeichnis
\definecolor{LinkColor}{rgb}{0,0,0.5}

% weitere Definitionen zum Aussehen und der Verwendung von Links 
\hypersetup{%
	colorlinks=true,%        Aktivieren von farbigen Links im Dokument (keine Rahmen)
	linkcolor=LinkColor,%    Farbe festlegen.
	citecolor=LinkColor,%    Farbe festlegen.
	filecolor=LinkColor,%    Farbe festlegen.
	menucolor=LinkColor,%    Farbe festlegen.
	urlcolor=LinkColor,%     Farbe von URL's im Dokument.
	bookmarksnumbered=true%  Überschriftsnummerierung im PDF Inhalt anzeigen.
}

 
% Definition der Counter für die Abschnittsnummerierung

% I Chapter
\renewcommand \thechapter {\Roman{chapter}}
% 1 Section 
\renewcommand \thesection {\arabic{section}}
% §1 Subsection
\renewcommand \thesubsection {\arabic{section}.\arabic{subsection}}
% §1 Subsection.1 Subsubsection
\renewcommand \thesubsubsection {§\arabic{subsubsection}}


% Definition der Schriften für die Überschriften 
\setkomafont{chapter}{\LARGE\bf}
\setkomafont{section}{\Large\bf}		
\setkomafont{subsection}{\large\bf}

% Setze die Anzahl der Unternummerierungen die möglich sind (z.B. 1.2.1.)
\setcounter{secnumdepth}{3}

\renewcommand{\theequation}{\arabic{equation}}

% Laden der Bibliographie
\usepackage[style=numeric,subentry]{biblatex}
% Lade alle BibTexeinträge unabhängig davon ob sie referenziert werden
% oder nicht
\nocite{*}
\bibliography{bib/cites.bib}



%==============================================================================
%		Einstellungen für Kopf- und Fußzeilen mit fancyhdr
%==============================================================================

% Definiere einen pagestyle 'fancy'
\pagestyle{fancy}
	
	% Setze alles zurück	
	\fancyhf{}
	 
	% Chapter wird nur in den  Dokumentklassen 'book' and 'report' verwendet 
	\renewcommand{\chaptermark}[1]{%
		\markboth{\chaptername \ \thechapter.\ #1%		
		}{}}
	
	\renewcommand{\sectionmark}[1]{%
		\markboth{#1 %
		}{}}
	\renewcommand{\subsectionmark}[1]{% 
		\markright{#1 %
		}{}} 
	
	% Definiere ob und wenn ja wie stark die Linien vor bzw. nach Kopf- 
        % und Fußzeile sind
	\renewcommand{\headrulewidth}{.5pt} 
	\renewcommand{\footrulewidth}{.5pt}
	
	% Einstellen der eigentlichen Information für die Kopfzeile
	\fancyhead[LE,RO]{\footnotesize{}}
	\fancyhead[CO]{\footnotesize{\rightmark}}
        \fancyhead[CE]{\footnotesize{\leftmark}}
	\fancyhead[RE,LO]{\footnotesize{}}
	 
	% Einstellen der eigentlichen Information für die Fußzeile
	\fancyfoot[LE,RO]{\footnotesize{}
	\fancyfoot[CE,CO]{\footnotesize{\thepage\,von \pageref{LastPage}}}}
	\fancyfoot[RE,LO]{\footnotesize{}}







%==============================================================================
%                     Definieren eigener Befehle
%==============================================================================

% Definiere den Befehl '\HRule' um eine 0.3mm dicke Haarline zeichnen zu können
\newcommand{\HRule}{\rule{\linewidth}{0.3mm}}


% Definiere die standardmäßig verwendeten mathematischen Zahlensymbole
%  (Verwendung des Paktes dsfonts) 
\DeclareMathOperator{\N}{\mathds{N}}			% natürliche Zahlen |N
\DeclareMathOperator{\No}{\mathds{N}\setminus\{0\}}	% natürliche Zahlen |N ohne die Null
\DeclareMathOperator{\Z}{\mathds{Z}}			% ganze Zahlen |Z
\DeclareMathOperator{\Q}{\mathds{Q}}			% rationale Zahlen |Q
\DeclareMathOperator{\R}{\mathds{R}}			% reelle Zahlen |R
\DeclareMathOperator{\C}{\mathds{C}}			% komplexe Zahlen |C
\DeclareMathOperator{\Hq}{\mathds{H}}			% Quaterninonen |H
\DeclareMathOperator{\F}{\mathds{F}}			% |F
\DeclareMathOperator{\LL}{\mathds{L}}			% |L
\DeclareMathOperator{\kk}{\mathds{k}}			% |k
\DeclareMathOperator{\PP}{\mathds{P}}			% |P
\DeclareMathOperator{\E}{\mathds{E}}			% |E



% Definition der mathematischen Operatoren für geschwungene Buchstaben

\DeclareMathOperator{\Aa}{\mathcal{A}}
\DeclareMathOperator{\Bb}{\mathcal{B}}
\DeclareMathOperator{\Cc}{\mathcal{C}}
\DeclareMathOperator{\Dd}{\mathcal{D}}
\DeclareMathOperator{\Ee}{\mathcal{E}}
\DeclareMathOperator{\Ff}{\mathcal{F}}
\DeclareMathOperator{\Gg}{\mathcal{G}}
\DeclareMathOperator{\Hh}{\mathcal{H}}
\DeclareMathOperator{\Ii}{\mathcal{I}}
\DeclareMathOperator{\Jj}{\mathcal{J}}
\DeclareMathOperator{\Kk}{\mathcal{K}}
\DeclareMathOperator{\Ll}{\mathcal{L}}
\DeclareMathOperator{\Mm}{\mathcal{M}}
\DeclareMathOperator{\Nn}{\mathcal{N}}
\DeclareMathOperator{\Oo}{\mathcal{O}}
\DeclareMathOperator{\Pp}{\mathcal{P}}
\DeclareMathOperator{\Qq}{\mathcal{Q}}
\DeclareMathOperator{\Rr}{\mathcal{R}}
\DeclareMathOperator{\Ss}{\mathcal{S}}
\DeclareMathOperator{\Tt}{\mathcal{T}}
\DeclareMathOperator{\Uu}{\mathcal{U}}
\DeclareMathOperator{\Vv}{\mathcal{V}}
\DeclareMathOperator{\Ww}{\mathcal{W}}
\DeclareMathOperator{\Xx}{\mathcal{X}}
\DeclareMathOperator{\Yy}{\mathcal{Y}}
\DeclareMathOperator{\Zz}{\mathcal{Z}}

% mathematischer Operator für Indikatorfunktionen mit bold-font
%\DeclareMathOperator{\1}{\mathbf{1}}

% mathematischer Operator für Indikatorfunktionen mit doppel Strich
\DeclareMathOperator{\I1}{\mathds{1}}

% verschiedene andere mathematische Operatoren 
\DeclareMathOperator{\Lp}{L}
\DeclareMathOperator{\Det}{det}
\DeclareMathOperator{\Map}{Map}
\DeclareMathOperator{\Mult}{Mult}
\DeclareMathOperator{\Alt}{Alt}
\DeclareMathOperator{\Mat}{Mat}
\DeclareMathOperator{\sgn}{sign}
\DeclareMathOperator{\Gl}{GL}
\DeclareMathOperator{\rk}{rk}
\DeclareMathOperator{\Bij}{Bij}
\DeclareMathOperator{\leit}{\textit{l}}
\DeclareMathOperator{\RT}{\mathnormal{R[T]}}
\DeclareMathOperator{\ggT}{ggT}
\DeclareMathOperator{\End}{End}
\DeclareMathOperator{\Hom}{Hom}
\DeclareMathOperator{\Invten}{Invten}
\DeclareMathOperator{\Id}{id}
\DeclareMathOperator{\cha}{char}
\DeclareMathOperator{\im}{im}
\DeclareMathOperator{\Tr}{Tr}
\DeclareMathOperator{\Bil}{Bil}
\DeclareMathOperator{\Aut}{Aut}
\DeclareMathOperator{\Orth}{O}
\DeclareMathOperator{\SO}{SO}
\DeclareMathOperator{\SL}{SL}
\DeclareMathOperator{\Sp}{Sp}
\DeclareMathOperator{\sk}{\mathnormal{\langle\, ,\,\rangle}}
\DeclareMathOperator{\SU}{SU}
\DeclareMathOperator{\Un}{U}
\DeclareMathOperator{\supp}{supp}

\DeclareMathOperator{\argmax}{\text{arg max}}
\DeclareMathOperator{\argmin}{\text{arg min}}
\DeclareMathOperator{\arginf}{\text{arg inf}}
\DeclareMathOperator{\argsup}{\text{arg sup}}


%==============================================================================
%				Einstellungen der theoremstyles
%==============================================================================

% Schreibe die Nummerierung vor die Bezeichnung
% 1.2 Satz. statt Satz 2.1. 
%\swapnumbers

% Standard theoremstyle für Sätze, Lemmata, usw.
\newtheoremstyle{plainn}% 	Name
{1.5pt}%      			Abstand zur Vorzeile (leer = default Wert)
{1.5pt}%      			Abstand zur Folgezeile
{\itshape}% 			Schriftart Body
{}%         			Einzug (leer = kein Einzug, \parindent = paragraph Einzug)
{\bfseries}% 			Schriftart Überschrift
{.}%        			Zeichen nach der Überschrift
{.5em}%     			Platz nach der Überschrift (\newline ruft einen Zeilenumbruch hervor)        							
{}% 				Thm head spec

% theoremstyle für Anmerkungen
\newtheoremstyle{anmerkung}% 	Name
{1.5pt}%      			Abstand zur Vorzeile (leer = default Wert)
{1.5pt}%      			Abstand zur Folgezeile
{}% 				Schriftart Body
{}%         			Einzug (leer = kein Einzug, \parindent = paragraph Einzug)
{}%	 			Schriftart Überschrift
{.}%        			Zeichen nach der Überschrift
{}%     			Platz nach der Überschrift (\newline ruft einen Zeilenumbruch hervor)        							
{}% 				Thm head spec


% Zuweisung der theoremstyles zu den einzelnen Umgebungen
% Umgebungen mit * sind unnummeriert


\addto\extrasngerman{%
  \def\satzautorefname{Satz}
  \def\lemmaautorefname{Lemma}  
  \def\theoremautorefname{Theorem}
  \def\korollarautorefname{Korollar}
  \def\propositionautorefname{Proposition}
  \def\beispielname{Beispiel}
  \def\definitionautorefname{Definition}
  \def\bemerkungautorefname{Bemerkung}
  \def\problemautorefname{Problem}
}

\theoremstyle{plainn}
  % verwende die Nummerierung von 'subsection' zusätzlich zur normalen Nummerierung
  \newtheorem{satz}{Satz}[section]
  \newtheorem*{satz*}{Satz}
  
  
  % verwende die selbe Nummerierung wie 'satz'
  \newaliascnt{lemma}{satz}
  \newtheorem{lem}[lemma]{Lemma}
  \newtheorem*{lem*}{Lemma}

  % verwende die selbe Nummerierung wie 'satz'
  \newaliascnt{theorem}{satz}
  \newtheorem{theorem}[theorem]{Theorem}
  \newtheorem*{theorem*}{Theorem}

  % verwende die selbe Nummerierung wie 'satz'
  \newaliascnt{korollar}{satz}
  \newtheorem{kor}[korollar]{Korollar}
  \newtheorem*{kor*}{Korollar}	

  % verwende die selbe Nummerierung wie 'satz'
  \newaliascnt{proposition}{satz}  
  \newtheorem{prop}[proposition]{Proposition}
  \newtheorem*{prop*}{Proposition}

  % verwende die selbe Nummerierung wie 'satz'
  \newaliascnt{problem}{satz}  
  \newtheorem{prob}[problem]{Problem}
  \newtheorem*{prob*}{Problem}


\theoremstyle{definition}
  % verwende die selbe Nummerierung wie 'satz'
  \newaliascnt{beispiel}{satz}
  \newtheorem{bsp}[beispiel]{Beispiel}
  \newtheorem*{bsp*}{Beispiel}   

  % verwende die selbe Nummerierung wie 'satz'
  \newaliascnt{definition}{satz}
  \newtheorem{Def}[definition]{Definition}
  \newtheorem*{Def*}{Definition}

  % verwende die selbe Nummerierung wie 'satz'
  \newaliascnt{bemerkung}{satz}
  \newtheorem{bem}[bemerkung]{Bemerkung}
  \newtheorem*{bem*}{Bemerkung}


\theoremstyle{remark} 
  \newtheorem*{bew*}{Beweis}


% spezielle Umgebung um eine 'leere Umgebung zu haben'
\theoremstyle{anmerkung}
  \newtheorem*{anm}{ }



%==============================================================================
%==============================================================================
%				Ende des Headers
%==============================================================================
%==============================================================================